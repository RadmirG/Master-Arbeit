\chapter*{Das inverse Problem}


In diesem Abschnitt soll der Bezug der Bildrekonstruktion aus CT-Daten zum Begriff \textit{inverser Probleme} hergestellt und der Frage nachgegangen werden, warum inverse Probleme meist schlecht gestellt seien. 

Zunächst können wir festhalten, dass in allen naturwissenschaftlichen Fragestellungen, die Wirkungen von den Ursachen herbei geführt werden. Dies ist eine Kausalität, welche sich in einem mathematischen Modell beschreiben lässt. Anlehnend an \cite[S. 14]{Rieder03} definieren wir ein mathematisches Modell, wie folgt

\begin{Definition}
	Ein \textbf{mathematisches Modell} ist eine Abbildung
	\[A: X \rightarrow Y\]
	von der Menge der Ursachen (Parameter) $X$ in die Menge der Wirkungen (Daten) $Y$. Im Falle des \textbf{inversen Problems} wird zu einer Wirkung $y \in Y$ die Ursache $x \in X$ gesucht, sodass $Ax = y$ erfüllt ist. Umgekehrt liegt ein \textbf{direktes Problem} vor.
	\label{def:3}
\end{Definition}
Jetzt ist es klar, warum Bildrekonstruktion aus CT-Daten zu den inversen Problemen gehört. Man möchte aus der Menge der Projektionsdaten $(Y)$ die unbekannte Dichteverteilung $(X)$ des bestrahlten Materials berechnen.

Im nächsten Schritt soll geklärt werden, was ein \textit{schlecht gestelltes} Problem ist. Dazu betrachten wir die Definition aus \cite[S. 24]{Rieder03}
\begin{Definition}
	Sei $A:X \rightarrow Y$ eine Abbildung zwischen Hilberträumen $X$ und $Y$. Das Problem $(A, X, Y)$ heißt \textbf{schlecht gestellt nach Nashed}\cite{Nashed} (ill-posed), wenn das Bild von $A$ nicht abgeschlossen ist. Ansonsten heißt das Problem $(A,X,Y)$ \textbf{gut gestellt nach Nashed}.	
	\label{def.4}
\end{Definition}
Nun ist es an der Zeit, die erarbeiteten Ergebnisse aus dem Kapitel (\ref{cha:1.1}) heranzuziehen. Entsprechend der Definition \ref{def.4}. schreiben wir den Ausdruck (\ref{equa:1.10}) als $(\mathcal{R}, L^2(\Omega), L^2(Z))$ um. Somit ist ein schlecht gestelltes Problem gegeben, welches durch die SWZ (\ref{equa:1.32}) des Operators $\mathcal{R}$
 
\[ \mathcal{R}f = \sum\limits_{j = 1}^{\infty} \sigma_j \langle f, v_j\rangle_{L^2(\Omega)} u_j. \]  

einzusehen ist. Man erkennt, dass das Bild des Operators $\mathcal{R}$ nicht abgeschlossen ist.

Wir wollen nun den Operator $\mathcal{R}$ invertieren. Aus dem \textit{Spektralsatz für selbstadjungierte kompakte Operatoren} \cite[S. 30]{Rieder03} folgt, dass die Eigenwerte von $\mathcal{R}\mathcal{R^*}$ der Bedingung

\[ \forall j \in \N : \lambda_j > 0 \ \ \mbox{und} \ \ \lambda_j \xrightarrow[{j \rightarrow \infty}]{} 0 \]

genügen. Entsprechend (\ref{equa:1.31}, $(i)$) sind die Singulärwerte von $\mathcal{R}$ alle größer Null. Dies erlaubt die Invertierung der Gleichung (\ref{equa:1.32}) und wir schreiben
\begin{equation}
	\mathcal{R}^{+}g = \sum\limits_{j = 1}^{\infty} \sigma_j^{-1} \langle g, u_j \rangle_{L^2(Z)} v_j \ \ \mbox{für} \ \ g \in \mbox{Bild}(\mathcal{R}).
	\label{equa:2.1}
\end{equation}
Aus der Gleichung (\ref{equa:2.1}) ist zu erkennen, dass der Ausdruck $\sigma_j^{-1}$ für  $j \rightarrow \infty$ gegen Unendlich läuft und somit ist $\mathcal{R}^{+}$ nicht beschränkt, was also die Nichtstetigkeit bedeutet. 

Die Bezeichnung $\mathcal{R}^{+}$ nennt man \textit{verallgemeinerte Inverse} des kompakten Operators $\mathcal{R}$ und wir bestätigen die obige Überlegung mit folgendem Satz, dessen Beweis in \cite[S. 32]{Rieder03} nachgelesen werden kann.
\begin{theorem}
	Sei $A : X \rightarrow Y$ ein kompakter Operator mit einem Singulärsystem $\{(\sigma_j, v_j, u_j)\}$. Dann ist die verallgemeinerte Inverse durch
	\[ A^{+} g  = \sum\limits_{j = 1}^{\infty} \sigma_j^{-1} \langle g, u_j \rangle_Y v_j \ \ \ f\ddot{u}r \ \ g \in \mathcal{D}_A \footnote{\label{foot:8}$\mathcal{D}_A$ : Definitionsbereich der Abbildung $A : X \rightarrow Y$.} \]
	gegeben. Hat $A$ ein endlichdimensionales Bild, dann ist $A^+$ stetig.   
	\label{satz:2}
\end{theorem}
Im Falle des Operators $\mathcal{R} : L^2(\Omega) \rightarrow L^2(Z)$ ist das Bild \textit{unendlichdimensional}, was auch dem Ausdruck (\ref{equa:1.32}) anzusehen ist. 

Gemäß obigen Überlegungen ist das Problem rein theoretisch. In der Praxis hängt die Radon Transformation nur von endlich vielen Argumenten ab, also von $s_i$ mit $i \in [1,k], \ k \in \N$ und  von $\theta_l$ mit $l \in [1,q], \ q \in \N$. Damit ist gemeint, dass man gerade $n = kq$ Projektionen erzeugen kann. Somit bekommen wir
\begin{equation}
	\mathcal{R}f = \sum\limits_{j = 1}^{n} \sigma_j \langle f, v_j\rangle_{L^2(\Omega)} u_j, \ \ n \in \N.
	\label{equa:2.2}
\end{equation}
\begin{Bemerkung}
	An dieser Stelle lassen wir die Diskretisierungstheorie unendlichdimensionaler Operatoren bei Seite. Wir verweisen hier auf \cite[S. 153]{Rieder03}. Zu erwähnen ist, dass $n \in \N$ in (\ref{equa:2.2}) nicht die ersten $n$ Eigenvektoren bezeichnet, sondern, dass die unendliche Dimension von $\mathcal{R}$ auf $n$-Dimensionalität interpoliert wurde.
	\label{bem:4}
\end{Bemerkung}
\begin{Bemerkung}
	Die Elemente $\mathcal{R}$, $f$, $\sigma_j$, $v_j$ und $u_j$ in (\ref{equa:2.2}) sind nicht mit denen aus (\ref{equa:1.32}) zu verwechseln. In diesem Falle sind es die diskretisierte Elemente.
	\label{bem:5}
\end{Bemerkung}
Wir bilden zuerst die Inverse $\mathcal{R}^{+}$ von $\mathcal{R}$ und diskutieren das Verhalten einzelner Summenglieder von
\begin{equation}
	\mathcal{R}^{+} g  = \sum\limits_{j = 1}^{n} \sigma_j^{-1} \langle g, u_j \rangle_{L^2(Z)} v_j.
	\label{equa:2.3}
\end{equation}
Hier stellt $g$ eine reale Messung der Projektionen dar, bei der gewisse Messungenauigkeiten anzunehmen sind. Diese Messungenauigkeiten werden mit steigendem $j$ um dem Faktor $\sigma_j^{-1}$ in Richtung des Eigenvektors $u_j$ verstärkt. Anderes ausgedrückt, wird auf diese Weise rekonstruierte Lösung $\mathcal{R}^{+} g$ immer verschwommener, was auf die Oszillation der Eigenvektoren zurückzuführen ist.

Im Allgemeinen bedarf es einer Bedingung an die Koeffizienten $\langle g, u_j \rangle_{L^2(Z)}$, die besagt, dass sie schnell genug abfallen müssen, um nicht von $\sigma_j^{-1}$ ins Unendliche geschickt zu werden. Genau das besagt die \textit{Picard-Bedingung} (s.a. \cite[S. 31]{Rieder03}).
\begin{theorem}[\textit{\textbf{Picard-Bedingung}}]
	Seien $X, Y$ Hilberträume und $A : X \rightarrow Y$ ein kompakter Operator mit einem singulärem System $\{(\sigma_j, v_j, u_j)\}$. Die Reihe 
	\begin{equation}
		\sum\limits_{j=1}^{\infty} \frac{|\langle g, u_j \rangle_Y|^2}{\sigma_j^2}
		\label{equa:2.4}
	\end{equation}
	konvergiert genau dann, wenn $g \in \mbox{Bild}(A)$.
	\label{satz:3}
\end{theorem}
Die verrauschten Messdaten liegen nie in dem Bild der Radon Transformation. Deshalb werden die Anteile der Lösung mit kleinen Singulärwerten (oder hochfrequente Anteile) mit speziellen Techniken \textit{gedämpft}, so dass die Reihe (\ref{equa:2.4}) konvergiert. Die Dämpfung hochfrequenter Anteile nennt man im Allgemeinen Regularisierung eines schlecht gestellten Problems. Die Art der Regularisierung unterliegt dem Rekonstruktionsverfahren. Das aber, wird das Hauptthema des nächsten Kapitels sein. 




