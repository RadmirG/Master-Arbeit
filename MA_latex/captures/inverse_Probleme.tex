\chapter{Inverse Probleme}
\label{cha:inverse_Probleme}

Das Problem der Bildrekonstruktion in Computertomographie lässt sich unter einer allgemeiner Problemklasse einordnen. In der Mathematik ist diese Klasse unter dem Namen \textit{inverse Probleme} bekannt. Ein inverses Problem liegt immer dann vor, wenn man aus einer beobachtbaren Wirkung eines Vorgangs auf dessen Ursache schließen möchte. Untersucht man in die Andere Richtung, so nennt das Vorgehen \textit{direktes Problem}.

Zunächst ist es wichtig festzulegen 

sei an dieser Stelle das abstrakte mathematische Modell anlehnend an \cite[Kap.\ 1.5]{Rieder03} betrachtet.
\begin{Definition}
	Ein \textbf{mathematisches Modell} ist eine Abbildung
	\[A: X \rightarrow Y\]
	von der Menge der Ursachen (Parameter) $X$ in die Menge der Wirkungen (Daten) $Y$. Im Falle des \textbf{inversen Problems} wird zu einer Wirkung $y \in Y$ die Ursache $x \in X$ gesucht, sodass $Ax = y$ erfüllt ist.
\end{Definition}
...
\begin{Definition}
	Sei $A:X \rightarrow Y$ eine Abbildung zwischen topologischen Räumen $X$ und $Y$. Das Problem $(A, X, Y)$ heißt gut gestellt wenn
	\begin{itemize}
		\item[(a)] Die Gleichung $Ax = y$ hat für jedes $y \in Y$ eine Lösung.
		\item[(b)] Die Lösung ist eindeutig bestimmt.
		\item[(c)] Die inverse Abbildung...
	\end{itemize}
\end{Definition}
