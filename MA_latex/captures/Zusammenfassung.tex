\section*{Eidesstattliche Erklärung} 
\thispagestyle{empty}

Hiermit erkläre ich, dass ich die vorliegende Bachelorarbeit selbstständig verfasst und nur unter
Zuhilfenahme der angegebenen Quellen und Hilfsmitteln angefertigt habe. Sämtliche Stellen der Arbeit, die im Wortlaut oder dem Sinn nach anderen gedruckten oder im Internet verfügbaren Werken entnommen sind, habe ich durch genaue Quellenangaben
kenntlich gemacht.\\
\vspace{20pt}\\
Berlin, \today\\
\vspace{15pt}\\
\par\noindent\rule{0.25\textwidth}{0.4pt}\\\\
Radmir Gelser


\newpage
\section*{Abstract} 	% engl. Preface

The present work contains a comprehensive discussion about the current reconstruction methods of computerized tomography (CT). In the first chapter, the mathematical-physical basics of CT will be discussed. Whereby much emphasis was placed on the understandable derivation of the mathematical model. Next, the basic properties of the underlying operator $\mathcal{R}$ of the Radon-Transform, especially the compactness of $\mathcal{R}$, are examined. In the second chapter the reference of the compact operators to the \textit{ill-posed} problems is shown. This relation leads to the statement that the CT-Reconstruction represents as an ill-posed problem. In the third chapter, concrete reconstruction techniques for computerized tomography image data are presented and investigated. Again, much emphasis was placed on their clean derivation from the Radon-Transformation, which should make the process transparent and understandable. In total, three methods were derived, the filtered backprojection, the iterative Kaczmarz method and the reconstruction by singular value decomposition.

\section*{Zusammenfassung}

Die vorliegende Arbeit stellt eine zusammenfassende Diskussion über die gängigen Rekonstruktionsverfahren der Computertomographie (CT) dar. Im ersten Kapitel werden zunächst die mathematisch-physikalischen Grundlagen der Computertomographie besprochen. Hierbei wurde viel Wert auf die verständliche Herleitung des mathematischen Modells gelegt. Dabei kommt man zu der Erkenntnis, dass das mathematische Modell der Radon Transformation entspricht, die mit dem Operator $\mathcal{R}$ bezeichnet wird. Anschließend werden die grundlegenden Eigenschaften des Operators $\mathcal{R}$ untersucht, insbesondere seine Kompaktheit. Im zweiten Kapitel wird der Bezug zwischen den kompakten Operatoren und den schlecht gestellten Problemen aufgezeigt. Dieser Bezug führt zu der Aussage, dass die CT-Rekonstruktion ein schlecht gestelltes Problem darstellt. Im dritten Kapitel werden konkrete Rekonstruktionstechniken für computertomographische Bilddaten vorgestellt und untersucht. Auch hier wurde viel Wert auf ihre übersichtliche Herleitung aus der Radon Transformation gelegt, was die Verfahren transparent und verständlich machen soll. Insgesamt wurden drei Verfahren hergeleitet, die gefilterte Rückprojektion, iteratives Kaczmarz-Verfahren und die Rekonstruktion durch Singulärwertzerlegung.